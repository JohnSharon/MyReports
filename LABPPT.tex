\documentclass{beamer}
\usetheme{default} % choose a theme
\usecolortheme{default} % choose a color scheme
\title{Raman Spectroscopy}
\author{S. John Sharon Sandeep}
\date{\today}

\usepackage{graphicx}
\usepackage{subcaption}
\usepackage{xcolor}

\usepackage{amssymb} %(ams symbols, use \mathbb{} for blackboard symbols)
\usepackage{amsmath}
\usepackage{amsthm}
\usepackage{cleveref}
\usepackage{listings}
\usepackage[shortlabels]{enumitem}

\setbeamertemplate{frametitle}{
    \nointerlineskip%
    \begin{beamercolorbox}[wd=\paperwidth,ht=2.5ex,dp=1.125ex]{frametitle}
        \centering
        \usebeamerfont{frametitle}\insertframetitle
    \end{beamercolorbox}
}

\begin{document}

\begin{frame}
\titlepage
\end{frame}

\begin{frame}
\frametitle{Outline}
\tableofcontents
\end{frame}

\section{History}
\begin{frame}
\frametitle{A New Type of Secondary Radiation}
\begin{figure}[ht]
  \centering
  \includegraphics[width=0.78\textwidth]{C:/Users/johns/Downloads/PPT/Ramanhimself.png}
    \caption{}
    \label{fig:1}
\end{figure}
\end{frame}

\section{Theory}
\begin{frame}
\frametitle{Quantum mechanical model of a diatomic molecule}
For a molecule approximated by two balls connected with a spring, added with a potential that accounts for:
\begin{itemize}[\textbullet]
\item repulsion between electronic clouds when the atomic nuclei approach
\item variable behaviour of the bond force when the atoms move apart from one another
\end{itemize}
the energy spectrum is:
\begin{itemize}[\textbullet]
\item discrete 
\item when written as an increasing function of vibrational quantum number $v$, the difference between two adjacent energy levels decreases with $v$
\item allows occurrence of transitions with $\Delta v \geq 2$
\end{itemize}
\end{frame}

\begin{frame}
Further if bond rupturing for over-displacement is accounted, the energy spectrum would look like figure B in the following:
\begin{figure}[ht]
  \centering
  \includegraphics[width=0.85\textwidth]{C:/Users/johns/Pictures/Screenshots/Screenshot (89).png}
  \caption{Schematic representation of the harmonic (A) and anharmonic (B) models for the potential energy of a diatomic molecule. $d_e=$ equilibrium distance (U $=$ minimum).}
  \label{fig:diatomicspectrum}
\end{figure}
\end{frame}


%\section{}
\begin{frame}
\frametitle{Possible transitions for different wavenumbers of incident light}
\begin{figure}[ht]
  \centering
  \includegraphics[width=0.85\textwidth]{C:/Users/johns/Downloads/PPT/JablonskiDiagram.png}
  \caption{Energy level diagram related to IR absorption, Raman scattering and fluorescence emission}
  \label{fig:jablonski}
\end{figure}
\end{frame}

\section{Raman Scattering}
\begin{frame}
\frametitle{Raman Scattering}
\begin{itemize}[\textbullet]
\item As seen in previous slide, a given molecule at one of its ground vibrartional states, when irradiated with a monochromatic light of wavenumber within a certain range (dependent on the molecule), say $\bar{\nu_s}$, it gets excited to a virtual state and within $10^{-14} s$, it returns to one of the ground vibrational state emitting a photon of the same energy it lost. Light due to such photons is called scattered light. This phenomenon is called Scattering of light. 
\item The scattered light is predominantly of the same wavenumber $\bar{\nu_s}$, and for some set of $\bar{\nu_0}s$, there is a very small intensity of light with wavenumber $\bar{\nu_s}+\bar{\nu_0}$ and of even less intesity with wavenumber $\bar{\nu_s}-\bar{\nu_0}$.\\
\textcolor{blue}{Inference/Reason?}
\item Defining ramanshift for a scattered light of wavenumber $\bar{\nu_0}$ as $\bar{\nu_s}-\bar{\nu_0}$ where $\bar{\nu_s}$ is the wavenumber of source light, lets see an example of raman spectrum for a molecule.
\end{itemize}
\end{frame}


\begin{frame}
\frametitle{Raman Spectra of CCl4}
\begin{figure}[ht]
  \centering
  \includegraphics[width=0.5\textwidth]{C:/Users/johns/Downloads/PPT/Raman-spectrum-of-CCl4-532nm-laser.png}
    \caption{Both anti-Stokes and Stokes lines (532nm source)}
    \label{fig:left}
\end{figure}
Where peaks corresponding to: 
\begin{itemize}[\textbullet]
\item{Positive Raman-Shifts: Stokes lines} 
\item{Negative Raman-Shifts:  Anti-Stokes lines} 
\item{0 Raman-Shift:  Rayleigh lines} 
\end{itemize}
\end{frame}


\begin{frame}
\begin{figure}[ht]
  \centering
  \includegraphics[width=0.85\textwidth]{C:/Users/johns/Downloads/PPT/ccl4.png}
    \caption{Stokes lines for CCl4 as obtained by our group in lab}
    \label{fig:right}
\end{figure}
\end{frame}

\section{Classical theory}
\begin{frame}
\frametitle{Classical theory}
Raman scattering originates from the interaction between incident electromagnetic radiation and 
molecular vibration. According to classical theory, when a molecule is  placed in an 
electric field $E=E_s \cos(2\pi \nu_{s} t)$, an electric dipole moment P is induced, 
\begin{equation}
P=\alpha E  \label{eq1}
\end{equation}
 where $\alpha$ is the polarizability. Considering only lower order Raman scattering, then since molecular vibrations are composed of several normal modes, it can be written as,
\begin{equation}
\alpha=\alpha_0+\bigg{(}\frac{\partial \alpha}{\partial Q_i}\bigg{)} Q_{i} + \ldots \label{eqn2}
\end{equation}
where $Q$ is the normal coordinates of the various vibration modes and can be expressed as, $Q_{i}=Q_{i0} \cos(2 \pi \nu_{i} t)$. Substituting \eqref{eqn2} into the dipole moment $P$ gives
\begin{equation}
P=\alpha_0 E_s \cos(2\pi \nu_{s} t) + \frac{1}{2} \bigg{(}\frac{\partial \alpha}{\partial Q_i}\bigg{)} Q_{i0} E_{s} [\cos(2\pi t(\nu_{s} +\nu_{i})) + \cos(2\pi t (\nu_0 - \nu_i))] + \ldots  \label{eqn3}
\end{equation}
\end{frame}


\section{Instrumentation}
\begin{frame}
\frametitle{Instrument setup}
\begin{figure}[ht]
  \centering
  \begin{subfigure}[b]{0.5\textwidth}
    \includegraphics[width=\textwidth]{C:/Users/johns/Downloads/PPT/dispersionspectro.png}
    \caption{Dispersion Raman Spectrometer}
    \label{fig:left}
  \end{subfigure}%
  \hfill
  \begin{subfigure}[b]{0.5\textwidth}
    \includegraphics[width=\textwidth]{C:/Users/johns/Downloads/PPT/FTspectro}
    \caption{FT-Raman Spectrometer}
    \label{fig:right}
  \end{subfigure}
  \caption{Schematic representations of experimental setups}
  \label{fig:both}
\end{figure}
\end{frame}

\begin{frame}
\frametitle{Instrumentation}
There are two main issues in Raman spectroscopy that need to be addressed by 
suitable instrumentation:
\begin{itemize}[\textbullet]
\item Inherent weakness of the Raman scattering signal as compared to the intense Rayleigh scattering
\item Spectral resolution
\end{itemize}

Five basic components for a Raman spectrometer:
\begin{itemize}[\textbullet]
\item An intense laser source
\item Sample handling unit 
\item Monochromator or interferometer 
\item Detector 
\item Signal processing and output device
\end{itemize}
\end{frame}




\begin{frame}
\frametitle{Instrumentation}
\large{\textcolor{blue}{Radiation Sources for Raman Spectroscopy}}
\\
\begin{itemize}[\textbullet]
\item Laser: high intensity, single wavelength and 
coherence
\item The intensity of Raman scattering is dependent on the fourth power of the frequency 
of the exciting radiation
\end{itemize}

\large{\textcolor{blue}{Sample Handling Devices}}
\\
\begin{itemize}[\textbullet]
\item Glass cuvettes can be used
\item Water is not as much of a problem as in IR spectroscopy
\end{itemize}

\large{\textcolor{blue}{Transducers or Detectors}}
\\
\begin{itemize}[\textbullet]
\item Initially eyes were the detectors followed by photographic plates
\item Photomultiplier tubes, FT instruments
\item Nowadays, including the ones in lab use charge-coupled devices (CCDs)
\item Filtering devices like holographic grating or 
interference filters kept before the detector to filter Rayleigh component
\end{itemize}


\end{frame}

\begin{frame}
\frametitle{Instrumentation}
\large{\textcolor{blue}{Fibre-optic probes}}
\\
\begin{itemize}[\textbullet]
\item Laser beam from source is focused onto a fine bundle 
of optical fibres called input fibres that transport it to the sample
\item These input fibres are enclosed by a set of optical fibres called collection fibres
\item Collection fibres collect the scattered radiation from the sample and send it back to the instrument
\item Directed to a suitable monochromator
\end{itemize}
\begin{figure}[ht]
  \centering
  \includegraphics[width=0.6\textwidth]{C:/Users/johns/Downloads/PPT/Ramanprobe.png}
    \caption{Optical path and measurement volume of Raman probe}
    \label{fig:1}
\end{figure}
\end{frame}



\section{Bibliography}
\begin{frame}
\frametitle{Bibliography}
\begin{itemize}[\textbullet]
\item IGNOU Self-Learning material (Unit-6 Raman spectroscopy)
\item Near Infrared Spectroscopy: fundamentals, practical aspects and analytical applications - Celio Pasquini
\item Raman Spectroscopy for In-Line Water Quality Monitoring - Instrumentation and Potential - Zhiyun, Deen M.J., Selvaganapathy
\end{itemize}
\end{frame}


\end{document}